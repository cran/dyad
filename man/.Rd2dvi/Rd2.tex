\nonstopmode{}
\documentclass[a4paper]{book}
\usepackage[ae]{Rd}
\usepackage{makeidx}
\makeindex{}
\begin{document}
\chapter*{}
\begin{center}
{\textbf{\huge \R{} documentation}} \par\bigskip{{\Large of \file{ojivemodel.Rd}}}
\par\bigskip{\large \today}
\end{center}

\HeaderA{ojivemodel}{Gottman-Murray Marriage Model with Ojive Influence}{ojivemodel}
\keyword{models}{ojivemodel}
\keyword{nonlinear}{ojivemodel}
\begin{Description}\relax
Fit the Gottman-Murray marriage model with the ojive influence
function, optimizing the threshold.
\end{Description}
\begin{Usage}
\begin{verbatim}
ojivemodel(weights, nregime = 3, mpr = NA)
\end{verbatim}
\end{Usage}
\begin{Arguments}
\begin{ldescription}
\item[\code{weights}] A data frame with two columns, one for the wife (person
1) and one for the husband (person 2) scores for each unit of time.
\item[\code{nregime}] Number of regimes (either 2 or 3). If 2 regimes are
specified, there is a negative and positive regime. If 3 regimes are
specified, there is a negative, neutral, and positive
regime. Default is 3.
\item[\code{mpr}] Minimum number of observations per regime. If not specified, set to 10\% of observations. 
\end{ldescription}
\end{Arguments}
\begin{Details}\relax
This function fits the following Gottman-Murray model of marriage,
where input is a series of discrete observations of the  wife (\eqn{W}{})
and the husband (\eqn{H}{}). The same model can apply to single-sex couples,
where the wife should be interpreted as partner 1 and the husband as
partner 2. 

\deqn{W_{t+1}= a_{0} + a_{1}W_{t} + I_{HW}(H_t)}{}

\deqn{H_{t+1}= a_{0} + a_{1}H_{t} + I_{WH}(W_t)}{}

In these equations, one partner exerts influence on the other as a
function of the previous timestep, denoted by the influence functions
\eqn{I_{HW}}{} (influence of husband on wife) or \eqn{I_{WH}}{} (influence
of wife on husband). 

The influence function here is an ojive function. This is based on the
assumption that the influence of one partner on the other is constant
until his/her behavior passes some critical value or threshold. The
region to each side of the threshold is referred to as a regime. The
ojive function used here is proposed by Hamaker, E., Zhang, Z., and
Van der Maas, H.L. (See References). 

For values in each regime, a horizontal line is fit to the influence.
By varying the
threshold values (one threshold if 2 regimes are used, and two
thresholds if 3 regimes are used) we can determine the fit with the
minimum sum of squares residuals. We can write the ojive influence
functions as \eqn{I_{HW}(H_t)=}{}

\deqn{\delta^{-}_w\quad \text{if} \quad H_t\le\tau^{-}_w}{l1 if H_t <= nth}
\deqn{0 \quad \text{if} \quad \tau^{-}_{w} < H_t \le \tau^{-}_w}{0
if nth < H_t}
\deqn{\delta^{-}_w  \quad \text{if} \quad \tau^{+}_{w} < H_t }{l1 if
pth < H_t}

Parameters are estimated simultaneously using the method described by
Hamaker, E., Zhang, Z., and Van der Maas, H.L. (See References).
\end{Details}
\begin{Value}
\code{ojivemodel} returns a list consisting of the fit for the
wife (person 1) and the husband (person 2). Each fit is an
object of class \code{ojivemodel}.
The \code{plot} method for \code{ojivemodel} objects graphs the
score against the influence and the interpolated influence
function. 
An object of class \code{ojivemodel} contains the
following parameters, depending on the number of regimes specified.
\begin{ldescription}
\item[\code{a0}] Initial state
\item[\code{a1}] Inertia
\item[\code{l1}] Difference in constant between neutral regime and negative
regime
\item[\code{l2}] Difference in constant between positive regime and neutral
regime (3 regime only)
\item[\code{th}] Threshold (2 regime only)
\item[\code{nth}] Threshold between negative and neutral regime (3 regime only)
\item[\code{pth}] Threshold between positive and neutral regime (3 regime only)
\item[\code{ss}] Sum squared residuals
\item[\code{loglik}] Log likelihood assuming equal variance of residuals
across regimes
\item[\code{nparams}] Number parameters, assuming unequal variance of
residuals across regimes
\item[\code{BICeq}] Bayesian Information Criterion calculated assuming equal
variance of residuals across regimes
\item[\code{BICneq}] Bayesian Information Criterion calculated assuming unequal
variance of residuals across regimes
\item[\code{nt}] Number of observations
\item[\code{nregime}] Number of regimes (2 or 3)
\item[\code{score}] Vector of scores
\item[\code{influence}] Vector of influence, calculated using a0 and a1 above
\end{ldescription}
\end{Value}
\begin{Author}\relax
Tara Madhyastha and Ellen Hamaker
\end{Author}
\begin{References}\relax
For a general description of the marriage model and influence functions, see
Gottman, J. M., Murray, J. D., Swanson, C., Tyson, R., \& Swanson, K. R. (2003). \emph{The Mathematics of Marriage: Dynamic Nonlinear Models}. The MIT Press.

The method of parameter estimation used here is described in
Hamaker, E., Zhang, Z., Van der Maas, H.L. Using threshold autoregressive models to study dyadic interactions. Psychometrika, in press.
\end{References}
\begin{SeeAlso}\relax
\code{\LinkA{bilinmodel}{bilinmodel}}, \code{\LinkA{combimodel}{combimodel}}, \code{\LinkA{origmodel}{origmodel}}
\end{SeeAlso}
\begin{Examples}
\begin{ExampleCode}
##---- Should be DIRECTLY executable !! ----
##-- ==>  Define data, use random,
##--    or do  help(data=index)  for the standard data sets.
require(dyad)
data(couple)
## fit an ojive model with 3 regimes
fit <- ojivemodel(couple, nregime=3);
## plot influence function for husband on wife
plot(fit$wife)

\end{ExampleCode}
\end{Examples}

\printindex{}
\end{document}
